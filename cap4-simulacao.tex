\section{Validação de projeto}
\label{sec:validacao}

Ao longo do texto, definiu-se os requisitos e como os atingi-los, porém ainda é
preciso estabelecer como ocorrerá a prova do cumprimento do que foi acordado.
Para isso, nessa seção serão propostos alguns testes que são capazes de validar
o cumprimento dos requisitos. Essa etapa de validação poderia ser executada
somente após o desenvolvimento do protótipo, contudo, quaisquer modificações
de projeto com o protótipo já finalizado acrescentaria em um grande custo de
financeiro e em atraso de projeto. Dessa forma, optou-se pela validação do
protótipo via simulação, o que permite testar o veículo em ambiente subaquático
simulado, sem risco de dano ao veículo e evitando grandes retrabalhos por
modificações no projeto. Nessa seção, será discutido como a simulação deve ser
realizada para validação dos requitos de projeto, assim como a configuração do
ambiente virtual submarino.

\subsection{Módulos de simulação}
\label{sec:simu-modules}

A simulação fidedigna de um veículo submarino deve contar com a simulação dos
atuadores, sensores, da física de um corpo rígido debaixo d'água e dos códigos
no veículo. Com o objetivo de obter uma simualação que incluisse todos esses
tópicos, algumas soluções de software disponíveis foram adotas e algumas
desenvolvidas, a especificação de cada uma e justificativa de escolha é o
objetivo desta seção.

\subsubsection{Sensores, atuadores e colisão}
O principal \it{software} a ser utilizado para simulação é o Gazebo \cite[gazebo].


\subsubsection{Hidrodinâmica e Hidroestática do Submarino}

\subsubsection{Comunicação Acústica}

\subsection{Missões}